% Created with jtex v.0.1.6
\documentclass{article}
\usepackage{hyperref}
\usepackage{datetime}
\usepackage{graphicx}
\usepackage{natbib}
\bibliographystyle{abbrvnat}

%%%%%%%%%%%%%%%%%%%%%%%%%%%%%%%%%%%%%%%%%%%%%%%%%%
%%%%%%%%%%%%%%%%%%%%  imports  %%%%%%%%%%%%%%%%%%%
\usepackage{url}
%%%%%%%%%%%%%%%%%%%%%%%%%%%%%%%%%%%%%%%%%%%%%%%%%%

% colors for hyperlinks
\hypersetup{colorlinks=true, allcolors=blue}

\newcommand{\logo}{
  \href{https://curvenote.com}{\includegraphics[width=2cm]{curvenote.png}}
}

\title{How I Stopped Worrying and Learned to Love Javascript}

\author{riziles}

\newdate{articleDate}{11}{2}{2023}
\date{\displaydate{articleDate}}

\begin{document}
\maketitle
\begin{center}\logo\end{center}


\subsection*{How I Stopped Worrying and Learned to Love Javascript}

So I love \href{https://jupyterbook.org/en/stable/start/your-first-book.html}{Jupyter Book}.
It's a great way to build a beautiful website with Python.
You can add beautiful interactive charts with one of the many plotting libraries supported by Python
such as \href{https://bokeh.org/}{Bokeh} or \href{https://plotly.com/python/}{Plotly}.

But if you want to make things REALLY interactive, you need to add some Javascript.
\href{https://panel.holoviz.org/user_guide/Links.html#defining-javascript-callbacks}{Holoviz Panel}
allows you to embed Javascript callbacks to charts and widgets which
can then be added to Jupyter Book. It's great! But after awhile, you end up learning so
much Javascript that you start wondering why you don't just do everything in Javascript.

But then you remember the beautiful LaTeX PDF documents you wrote created with Jupyter Book,
and you decide to stick with Python.

But then you learn about \href{https://myst-tools.org/docs/mystjs}{Myst JS}! And you think,
what if I build a website in Javascript, and use Myst JS to convert it to PDF!


\bibliography{main.bib}
\end{document}
